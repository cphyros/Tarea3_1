Integrantes:

- Francisco Morales
- Joaquín Contreras

1. Exportar los archivos del proyecto a un IDE.

2. Colocar las 4 clases java dentro del mismo package.

3. Ejecutar el archivo testHashingDoble.java

Métodos implementados para el Hashing Doble:

- HashingDoble(int)	                        Constructor que define si la tabla hash si tiene de tamaño un número primo
                                                y en el caso que no lo tenga lo modifica.

- insert(movil) 	                              Inserta un objeto de la clase movil y se define su posición.

- find(int) 		                        Encuentra un elemento en la tabla.

- displayTable()	                              Imprime en consola.

- Trayectoria(int)	                        Imprime la trayectoria del movil ingresado en tiempo O(l)

- Distancia(int)                                Método que navega por la trayectoria del movil ingresado guardando sus
                                                posiciones en tiempo O(l)
                                                Termina imprimiendo la suma de todas sus posiciones

- PosiblementeCercanos(movil, movil, int)       Método que permite verificar si ambos móviles posiblemente estuvieron cercanos.        

- dist(coordenada,coordenada)                   Método privado para calcular la distancia entre dos puntos

- esPrimo(int)   	                              Verifica si el tamaño del array es de un número primo.

- getNextPrime(int) 	                        Si un array es de un tamaño no primo lo cambia al tamaño del número primo más cercano.

- hashFunc1(int)	                              Calcula la posición ideal del elemento en la tabla.

- hashFunc2(int)	                              Retorna el tamaño que tendrá el salto si llega a haber una colisión.